\paragraph{KML}\cite{wiki:kml}
\en{Keyhole Markup Language} est une notation XML pour exprimé les annotations geographic et les visualisation des catographies bi-dimentionel et tri-dimentionel dans un explorateur. \aprv{KML} a ete developper pour \en{Google Earth} et à été crée par \en{Keyhole. Inc} qui à été acquit par \en{Google} en 2004. le \en{Open Géospatial Consortium} l'a rendu un standard international en 2008. \en{Google Earth} a éte le premier programme capable de visualisé et modifier graphiquement les fichiers \aprv{KML}. Autre projets comme \en{Marble} commence a supportés \aprv{KML}.
\paragraph{GPX}\cite{wiki:gpx}
\en{GPX}
\en{GPS eXchange Format} est un schema XML designé comme un format de données \aprv{GPS} commun pour les applications logiciel.
IL peut étre utilisé pour décrire des \en{waypoints}, trajets et routes. Le format est ouvert et peut étre utilisé sans frés. Ces tags permet de sauvgardé le lieu, l'elevation et le temps ce qui permet d'echangé ces donnée entre les terminales GPS et les logiciels. Ces programme permet a l'utilisateur, par exemple, d'visualisé leur trajet, projeté ces trajet sur les images satellites ou autre cartes, annoter les cartes, et marqué les photographies avec la géolocalisation dans les metadata \aprv{Exif}.