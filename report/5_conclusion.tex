%!TEX root = report.tex

\chapter{Conclusion Général}   

L'intégration des technologies au sein des établissements médicaux est , malgré
les divers obstacles, une tendance établie et représente un marché juteux pour
les sociétés désirant le conquérir. Justifiant la judicieuse idée derrière ce
projet.

Reste que l'application en elle même reste limité. En particulier, le processus
de déploiement  suggère un minimum d'infrastructures requise, donc pour offrir
l'expérience désirer une solution alternative de support développer par Tunav
est de rigueur pour soit comblé le manque dans les équipements de
l’ établissement client ou dans le cas extrêmes les supplanté. Une stratégie de
commercialisation et un besoin évidant.

Ce projet peut être qualifié de type \en{proof of concept}, qui vise à explorer
une idée et vérifier son applicabilité. Une aubaine pour l'application produite
qui, en tout honnêteté, n'est pas encore au point et souffre de plusieurs
lacunes de conceptions et d'implémentation. Si un produit sérieux dans le même
thème est à offrir par Tunav, des efforts de recherche et de développement sont
de mise. En particulier l'intégration de médecins pratiquants dans des hôpitaux
au processus de conception et de test serai critique pour la compétitivité du
produit.

Cependant, les problèmes techniques pour le développement de cette application
ne sont pas les seuls à freiner sont adoption. Outre le problème de coûts  et
l'effort de persuasion requit, c'est un problème d'ordre psychologique qu'il
faut y faire face. En effet, avec tout concepts qui change radicalement des
procédures bien établie, un réticence de la par des utilisateurs ciblés , en
occurrence les médecins et le staff médical dans un contexte plus large, risque
de saboté les tests d\'intégrations. Des compagne de sensibilisation sont à
prévoir.