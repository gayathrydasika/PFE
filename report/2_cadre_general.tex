%!TEX root = report.tex

\chapter{Cadre Général du Projet}
\section{Introduction du chapitre}
Ce chapitre est subdivisé en deux partis: la première parti est consacré pour à la présentation de l'organisme d'accueil \textit{Tunav}. La deuxième parti est destiné à la présentation du projet en soit et les différents facteurs qui on pesés l'or du passage vers la réalisation.

\section{Présentation de l'organisme d'accueil}  

Tunav se situ à la Cité Technologique des Communications Parc Technologique
-BP-55 LA GAZELLE 2088 ARIANA, et a était fondé par sont Président Directeur
Général Mohamed Anis Kallel.

En guise de présentation, rien de mieux que de l'avoir directement du boss lui
même\cite{index_tunisie}:

"Tunav est une société technologique, créée au mois d'août 2004, implantée à la
technopole El Gazala et spécialisée dans la technologie GPS et ses diverses
applications dans les domaines de navigation et de gestion de flotte."

"Tunav est connue en Tunisie par son système \og{}LaTrace\fg{} de gestion de
flotte par GPS, lequel a été commercialisée pour la première fois en Octobre
2005. Il s'agit d'un système articulé autour d'une application très évoluée de
gestion de flotte, d'une gamme d'appareils GPS/GPRS et d'une base de données
géographique richement renseignée."

Tunav possède un savoir faire reconnu dans le domaine de la localisation qui peux être exploité dans le domaine médical.

\section{Présentation du projet}

\subsection{Utilisateurs Cibles}

Cette application vise \underline{principalement} les médecins. Et malgré que,
suite à des choix conceptuels, rien n'empêche qu'avec des modifications minime
une audience plus large dans le corps médical pourrai être ciblées, ce n'ai pas
-pour le moment- le but de l'application. Les médecins, malgré leur formation
prolongé dans le domaine médicale, représente une cible sans une vrais
profondeur technique, ce que requière de l'application d'être le plus simple
possible.

\subsection{Besoins fonctionnels}
\begin{itemize}

\item Le médecin doit être capable à partir de son terminal d'avoir des
informations sur les patients qui lui sont assigné en fonction de leur position
géographique.

\item L'application doit être capable de détecté la proximité d'un patient en fonction de la position du terminal.

\item Le médecin peut télé-consulter et accéder au dossier médical du patient.

\item Le médecin peut effectuer des maintenances des données contenu dans les
dossiers médicaux des patients.

\end{itemize}

\subsection{Besoins non fonctionnels}
\begin{itemize}

\item Une bonne ergonomie qui vise à faciliter l'obtention de l'information,
avec un minimum d'efforts pour l'utilisateur cible et avec le moindre risque
d'erreur. Les choix graphiques et conceptuels sont des considération à tenir
en compte.

\end{itemize}

\subsection{Besoins techniques}
\begin{itemize}

\item L'application mobile vise à utilisé les systèmes déjà en place des
établissements clients. Vu l'absence de standardisation et les différentes
implémentation possible, une certaine abstraction est requise pour pouvoir
déployer l'application dans des environnements possible avec le minimum de
modification.

\end{itemize}

\section{Conclusion du chapitre} 

La présentation de l'entreprise nous à permit de mieux cerné les points forts
qu'on pourrait compté sur pendant le développement de notre solution. Et une
connaissance exhaustive des objectifs de ce projet offre une base solide
nécessaire pour éviter de s’engager dans des fausses pistes.
