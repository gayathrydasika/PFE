%!TEX root = report.tex

\chapter{Introduction Général}

Ces temps ci, le mobile s’est imposé et devient la norme pour les
consommateurs. Les statistiques ne le cachent pas, c’était prévisible,
mais tous les analystes le soulignent: "Le marché des PC s’effondre face
aux smartphones et aux tablettes"\cite{lefigaro}. Et un des secteurs qui
pourrait bénéficier de l'avalanche des systèmes mobile est le secteur
médical.

Les applications mobiles offrent un potentiel énorme pour supporter et
activer des nouvelles opportunités pour les services médicaux.
Localisation, instantanéité, efficacité, personnalisation et une très
grande accommodation vont offrir plusieurs moyens nouveaux pour
améliorer l’expérience des services médicaux, du côté du patient
sûrement, mais tend aussi à rendre l’établissement plus convivial pour
les médecins et en général le staff médical.

Investir dans une application mobile représente pour les hôpitaux, et
les institutions qui les implémentent, un autre moyen pour étendre les
outils numériques déjà en place, en offrant des fonctionnalités qui sont
auparavant cloué aux ordinateurs des administrations. Ceci  facilitera
le processus de traitement des malades.

Cependant, l’usage des smartphone dans les établissements est sujet aux
questions notamment sur le plan technique. Les technique d’accès et de
sécurisation des données des patients et divers technologies utilisés et
surtout le manque de standardisation pose un sérieux challenge pour les
entreprises voulant offrir des solutions pour les établissements
médicaux.

Dans ce même thème se présente ce projet de fin l’étude sur la
conception et développement d’une application mobile sur plate-forme
Androïd destiné aux médecins dans le but de faciliter l’accès aux
dossiers médicaux des patients en intégrants les techniques de
localisation. Ce rapport est subdivisé en trois parties: La première
partie expose le cadre général du projet en présentant l’entreprise hôte
ainsi que les objectifs de l’application. La deuxième partie évoque les
solutions similaires déjà présentes dans le marché ainsi que une
présentation de la plate-forme sur la quelle l’application est à
développer. La troisième et dernière partie décortique le travail
effectué pour accomplir les objectifs.
