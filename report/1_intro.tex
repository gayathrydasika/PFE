%!TEX root = report.tex

\chapter*{Introduction Générale}
\addcontentsline{toc}{chapter}{Introduction Générale}

Ces temps-ci, le mobile s’est imposé et devient la norme pour les consommateurs.
Les statistiques ne le cachent pas, c’était prévisible, mais tous les analystes
le soulignent : "Le marché des PC s’effondre face aux smartphones et aux
tablettes" \cite{venturebeat.com}. Un des secteurs qui pourrait bien bénéficier
de l’avantage des systèmes mobiles est le secteur médical. Les applications
mobiles offrent un potentiel énorme pour supporter et activer des nouvelles
opportunités pour les services médicaux. La localisation, l’instantanéité,
l’efficacité, la personnalisation et une très grande accommodation vont offrir
plusieurs moyens nouveaux pour améliorer l’expérience des services médicaux, du
côté du patient sûrement, mais tendent aussi à rendre l’établissement plus
convivial pour les médecins et en général, le staff médical. Investir dans une
application mobile représente pour les hôpitaux, et les institutions qui les
implémentent, un autre moyen pour étendre les outils numériques déjà en place,
en offrant des fonctionnalités qui sont auparavant clouées  aux ordinateurs des
administrations. Ceci facilitera le processus de traitement des malades.
Cependant, l’usage des smartphones dans les établissements soulève des
questions, notamment sur le plan technique. Les techniques d’accès et de
sécurisation des données des patients et divers technologies utilisées, surtout
le manque de standardisation, posent un sérieux challenge pour les entreprises
voulant offrir des solutions pour les établissements médicaux. Dans ce même
thème se présente ce Projet de Fin d’Etudes sur la conception et le
développement d’une application mobile sur plate-forme Android destinée aux
médecins dans le but de faciliter l’accès aux dossiers médicaux des patients en
intégrant les techniques de localisation.

Ce rapport s'articule comme suit : La première partie expose le cadre
général du projet en présentant l’entreprise hôte ainsi que les objectifs de
l’application. Ensuite la deuxième partie évoque les solutions similaires déjà présentes
dans le marché ainsi qu’une présentation de la plate-forme sur laquelle
l’application est à développer. Après on enchaîne avec les études des besoins, de la structure, du comportement, ainsi que le déploiement et de teste de l'application. En fin on termine par une conclusion générale.
