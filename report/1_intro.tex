%!TEX root = report.tex

\chapter{Introduction Général}

Ces temps ci, le mobile s'est imposé et devient la norme pour les
consommateurs. les statistiques ne le cache pas, c'était prévisible, mais tout
les analystes le soulignes: "Le marché des PC s'effondre face aux smartphones
et aux tablettes"~\cite{lefigaro}. Et un des secteur qui pourrait bénéficier
par l'avalanche des systèmes mobile est le secteur médicale.

les applications mobiles offre un potentiel énorme pour supporté et activé des
nouvelles opportunité pour les services médicaux sont impressionnante.
Localisation, instantanéité, efficacité, personnalisation et une très grande
accommodation vont offrir plusieurs moyen nouveau pour amélioré l'expérience des
services médicaux, du côté du patient serte, mais tend aussi à rendre
l'établissement plus conviviale pour les médecins et en général du staff
médicale.

Investir dans une application mobile de représente pour les hôpitaux ,et les
institutions qui les implémentes, un autre moyen pour étendre les outils
numériques déjà en place. En offrent des fonctionnalités qui sont auparavant
cloué aux ordinateurs des administrations, ce qui facilite le processus de
traitement des malades.

Ce pendant, l'usage des smartphone dans les établissement est sujet aux
questions notamment sur le plan technique. les technique d'accès et de
sécurisation des données des patients et divers technologies utilisé et surtout
le manque de standardisation pose un sérieux challenge pour les entreprises
voulant offrir des solutions pour les établissements médicaux.

Dans ce même thème se présente ce projet de fin l'étude sur la conception et
développement d'une application mobile sur plate-forme \android destiné aux
médecins dans le but de facilité l'accès au dossiers médicaux des patients en
intégrants les techniques de localisation. Ce rapport est subdivisé en trois
sous parties: La premier parti expose le cadre général de projet en présentant
l'entreprise hôte ainsi que les objectifs de l'application. La deuxième partie
évoque les solutions similaires déjà présente dans le marché ainsi que une
présentation de la plate-forme pour la quelle l'application est développer. La
troisième et dernière parti décortique le travail effectué pour accomplir les
objectifs.
