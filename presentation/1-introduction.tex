%!TEX root = presentation.tex
\section{Introduction}
\subsection{Android}

\begin{frame}{Présentation}%FIXME
%%% HISTORIQUE
\only<+>{
	\begin{figure}
        \flushleft
        \pgfuseimage{android-mascot-2} 
    \end{figure}
}
%%% MARKET SHARE AND POPULARITY
\end{frame}

\begin{frame}{Développement dans un environnement mobile Android} %FIXME
%%% 
%Régle de si un processus est irrosponsive pandants 5s...
\end{frame}

\subsection{TunavMedi}%FIXME

\begin{frame}{Tunav}

\end{frame}

\begin{frame}{TunavMedi}
\only<+>{%%%LOGO
    
}

\only<+>{
\begin{block}{}
\begin{itemize}
\item Application Android.
\item Déstiné aux etablissements medicaux (Hopitaux, Cliniques, \dots).
%Speech: donc vous ne la trouvré pas dans le play store...
\item Gestion du workflow des medecins (taches à effectuées, dossiers medicaux \dots). %FIXME
\item \alert{Integration des services de localisation.}
\end{itemize}
\end{block}
}

%%% interactions
\only<+>{
\begin{figure}
    \centering
    \pgfuseimage{scenario-interactions}
\end{figure}
}
\end{frame}

\begin{frame}{Scénarios de déploiement}
%%% SCENARIO 1
\only<+>{
\inTitle{Scénario 1}
Tunav fourni le terminal et l'infrastructure.
\begin{figure}
    \centering
	\pgfuseimage{scenario-alltunav}
\end{figure}
}

%%% SCENARIO 2
\only<+>{
\inTitle{Scénario 2}
Integration dans une infrastructure déja existante.
\begin{figure}
    \centering
	\pgfuseimage{scenario-other}
\end{figure}
}
\end{frame}
