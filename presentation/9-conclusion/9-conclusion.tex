% !TEX root = ../0-presentation.tex
\section{Conclusion \& Perspectives}

\begin{frame}{Conclusion}
%%% THE GOOD
\only<+>{
	\begin{block}{Ce qu'on a aimé}
	
	\end{block}
}

%%% THE BAD
\only{
	\begin{block}{Ce qu'on peut amélioré}
	
	\end{block}
}

%%% THE UGLY
\only{
	\begin{block}{Ce qu'on aurai dû faire}
	
	\end{block}
}
\end{frame}

\begin{frame}{Perspectives}
\only<+>{
	\centerline{4 mois c'est un peut court pour un projet...}
	%Speech: ce qui peut en parti justifié les travaux médiocre présenté par les éleves...
}
\only<+>{
	\begin{itemize}
	\item Perspectives immédiat.
	\item Perspectives future.
	\end{itemize}
}
\end{frame}

\begin{frame}{Perspectives Immédiat}

\end{frame}

\begin{frame}{Perspectives Future}
\only<+>{
	Intégration à un (des) système(s) d'Information de gestion des hôpitaux et services de santé.
	\begin{itemize}
	\item GNU Health.
	\end{itemize}
}
\only<+>{
	Système d'identification des patients admit
	%%%Speech
	% Un des goulot pour les systèmes de santé est l'admission des 
	% nouveaux patients. Particulièrement le comment notifier les parties concerné par l'arrivé de ce patient dans l’établissement, exemple si le malade à besoin d'un scan RMI, il devrait étre automatiquement reconnu quant il se présente devant le service en question.
	\begin{itemize}
	\item NFC?
	\item QRcode?
	\end{itemize}
}
\end{frame}

\section*{Références \& Remerciement}
\begin{frame}
%%%Speech
%Donc voyons que vous êtes encore là, et éveiller en plus! ^_^...
\centerline{Merci!}
\end{frame}